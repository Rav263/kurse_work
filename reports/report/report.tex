\documentclass[a4peper, 12pt, titlepage, finall]{extreport}

%различные пакеты

\usepackage[T1, T2A]{fontenc}
\usepackage[russian]{babel}
\usepackage[backend=bibtex]{biblatex}
\usepackage{csquotes}
\usepackage{tikz}
\usepackage{geometry}
\usepackage{indentfirst}
\usepackage{fontspec}
\usepackage{graphicx}
\usepackage{array}
\graphicspath{{./images/}}

\usetikzlibrary{positioning, arrows}

\geometry{a4paper, left = 15mm, top = 10mm, bottom = 15mm, right = 15mm}
\bibliography{report}
\setmainfont{Spectral Light}%{Times New Roman}
\setmonofont{Courier New}
\setcounter{secnumdepth}{0}
%\setcounter{tocdepth}{3}
\nocite{*}
\begin{document}
\begin{center}
    {\large \bf Отчет о научной работе за пятый семестр по курсовой работе по теме:
    «Анализ и исследование структур данных и алгоритмов поиска в архитектуре сетевого процессора без выделенного ассоциативного устройства.»}

\end{center}
        \begin{flushright}
            {Никифоров Никита Игоревич, 321 группа}\\
            {Научные руководители:\\ Волканов Д. Ю., Скобцова Ю. А.}
        \end{flushright}
    \section{Введение}
        В настоящее время активно развивается технологий программно-конфигурируемых сетей (ПКС), в которых требуются высокопроизводительные 
        коммутаторы. Возникает задача разработки программируемого сетевого процессора, являющегося основным элементом комутаторов. Сетевой процессор 
        представляет из себя интегральную микросхему, специализированную для обработки сетевых пакетов, которая выполняет следующие функции:
        получение пакета с физического порта, выделение заголовка,
        классификация пакета по его заголовку, принятие решения о дальнейшем пути следования пакета, отправка пакета на физический порт~\cite{chao2007high:1}.
        В настоящее время активно ведётся разработка программируемых сетевых процессоров, то есть сетевой процессор позволяет менять
        программу обработки пакетов и набор различаемых полей заголовков.
        
        Исходя из функций сетевого процессора целесообразно рассматривать архитектуру основанную на
        наборе конвейеров, которая позволяет с фиксированной задержкой обрабатывать каждый пакет. 
        Конвейер в сетевом процессоре состоит из вычислительных блоков. В своей работе я рассматривал стадию классификации пакетов. Классификация $-$ процесс поиска в таблицах классификации
        по признакам пакета. Таблица классификации $-$ набор правил содержащих в себе признаки, по которым идентифицируется группа пакетов,
        и действия, которые сетевой процессор выполняет над данной группой пакетов. 
        Таким образом, для выполнения классификации сетевой процессор должен включать в себя ассоциативное устройство. Для реализации этого устройства естественным будет использование 
        ассоциативной памяти. Один контроллер ассоциативной памяти будет являться узким местом, так как к нему должны иметь доступ все стадии конвейера.
        Соответственно возникает потребность усложнения архитектуры, путём добавления в неё нескольких контроллеров ассоциативной памяти.
        Чтобы избежать сложной организации памяти в архитектуре можно отказаться от использования ассоциативной памяти. 
        В таком случае одно из решений $-$ совместить память команд и данных, и разместить память на кристалле.

        Таким образом, возникает задача разработки структур данных для поиска в таблицах классификации в сетевом процессоре без выделенного ассоциативного устройства.
    \section{Постановка задачи}
        Необходимо разработать структуру данных для поиска в таблицах классификации в рамках архитектуры сетевого процессора без выделенного ассоциативного устройства. 
        \\
        Разрабатываемая структура данных должна удовлетворять следующим требованиям.
        \begin{enumerate}
            \item Поиск по структуре данных не должен превышать 250 тактов сетевого процессора.
            \item Объём памяти занимаемый структурой данных не должен превышать 2 МБ.
            \item Реализуемая структура данных должна быть универсальна, а именно должна позволять выполнять поиск любых битовых строк длиной до 128 бит.
        \end{enumerate}
    \section{Проделанная работа}
        \begin{itemize}
            \item Изучена предметная область, а именно выполнены tutorial-ы по openFlow и P4, прочитанная литература про применение сетевых процессоров.
            \item Изучена рассматриваемая архитектура и эмулятор данной архитектуры сетевого процессора.
            \item На базе эмулятора были реализованны простейшие структуры данных, а именно бинарное однобитное дерево, сжатое бинарное однобитное дерево.
            \item Были реализованы утилиты для упрощения дальнейшей реализации структур данных для поиска в таблицах классификации, и их дальнейшего экспериментального исследования.
            \item Был проведён обзор структур данных для поиска в таблицах классификации в рамках рассматриваемой архитектуры сетевого процессора.
        \end{itemize}
    \section{Дальнейшая работа}
        \begin{itemize}
            \item Реализовать выбранные, в рамках проведённого обзора, структуры данных на эмуляторе сетевого процессора.
            \item Провести оптимизацию реализованных структур данных под рассматриваемую архитектуру сетевого процессорного устройства.
            \item Провести эксперементальное исследование реализованных структур данных.
        \end{itemize}
        \begingroup
        \let\clearpage\relax
        \printbibliography
        \endgroup
\end{document}
