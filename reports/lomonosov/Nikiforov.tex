\documentclass[usePics]{lomonosov}
 
\begin{thesis}  % Сам тезис должен быть полностью помещен внутри окружения thesis
  
% Один автор
\Title{Анализ и исследование структур данных для поиска в таблицах
классификации в архитектуре сетевого процессора без выделенного
ассоциативного устройства}{{Никифоров\,Н.\,И.}} 

\Author{Никифоров~Никита~Игоревич}{Студент}{Факультет ВМК МГУ имени М.\,В.\,Ломоносова}{Москва}{Россия}{nickiforov.nik@gmail.com}{Волканов~Д.Ю., Скобцова~Ю.А.}

    В настоящее время активно развивается технология программно-конфигурируемых сетей
    (ПКС), в которых требуются высокопроизводительные коммутаторы~[1]. Возникает задача разработки
    программируемого сетевого процессора, являющегося основным элементом коммутаторов.
    В данной работе будет рассматриваться проблема классификации пакетов в рамках архитектуры сетевого процессора без выделенного
    ассоциативного устройства памяти. Под классификацией понимается процесс идентификации пакета по его заголовку.
    Для этапа классификации в рамках рассматриваемой архитектуры сетевого процессора требуется реализация структур данных для хранения таблиц классификации.
    В качестве критериев выбора структур данных были минимизация используемой памяти и минимизация времени классификации. Был проведён обзор
    существующих структур данных, в котором были учтены ограничения рассматриваемой архитектуры сетевого процессора, а именно
    ограничение тактов сетевого процессора на обработку одного пакета, объём доступной памяти и отсутствие адресуемой памяти.
    В обзоре рассматривались только древовидные структуры данных~[3].
    На основе обзора было выбрано АВЛ дерево, с использованием алгоритма представления префиксов, как скалярных величин~[4], 
    для дальнейшей адаптации под архитектуру сетевого процессора. 
    Для проведения экспериментального исследования адаптированная структура данных была реализована на эмуляторе сетевого процессора.
    Использование АВЛ дерева позволило сократить объём используемой памяти сетевого процессора, а также количество тактов процессора на классификацию пакета.



\begin{references}
\Source Семелянский\,Р.\,Л. Програмно-конфигурируемые сети:
    \url{https://www.osp.ru/os/2012/09/13032491/}

\Source \ENGLISH{Chao, H. Jonathan, and Bin Liu. High performance switches and routers. John Wiley & Sons, 2007.}

\Source \ENGLISH{Le, Hoang, and Viktor K. Prasanna. "Scalable tree-based architectures for IPv4/v6 lookup using prefix partitioning." IEEE Transactions on Computers 61.7 (2011): 1026-1039.}

\Source \ENGLISH{Behdadfar, Mohammad, et al. "Scalar prefix search: A new route lookup algorithm for next generation internet." IEEE INFOCOM 2009. IEEE, 2009.}

\end{references}

\end{thesis} % Сам тезис должен быть полностью помещен внутри окрежения thesis

 
