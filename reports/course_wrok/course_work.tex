\documentclass[a4peper, 12pt, titlepage, finall]{report}

%различные пакеты

\usepackage[T1, T2A]{fontenc}
\usepackage[russian]{babel}
\usepackage[backend=bibtex]{biblatex}
%\usepackage{booktabs}
\usepackage{tikz}
\usepackage{geometry}
\usepackage{indentfirst}
\usepackage{fontspec}
\usepackage{graphicx}
\usepackage{array}
\graphicspath{{./images/}}

\usetikzlibrary{positioning, arrows}

\geometry{a4paper, left = 15mm, top = 10mm, bottom = 20mm, right = 15mm}

\setmainfont{Spectral Light}%{Times New Roman}
\setmonofont{Courier New}
%\setcounter{secnumdepth}{3}
%\setcounter{tocdepth}{3}

\bibliography{course_work} 

\begin{document}

    \begin{titlepage}
        \begin{center}
            {\small \sc Московский государственный университет имени М.~В.~Ломоносова\\
            Факультет вычислительной математики и кибернетики\\
            Кафедра автоматизации систем вычислительных комплексов\\}
            \vfill
            {\large \sc Курсовая работа}\\~\\

            {\large \bf Анализ и исследование структур данных для поиска в таблицах классификации в архитектуре сетевого процессора без выделенного ассоциативного устройства}\\~\\

        \end{center}
        
        \begin{flushright}
            \vfill
            \vfill
            {Никифоров Никита Игоревич, 321 группа}\\
            {Научные руководители: к. ф.-м. н., доцент Волканов Д. Ю., Скобцова Ю. А}
        \end{flushright}

        \begin{center}
            \vfill
            {\small Москва\\2020}
        \end{center}
    \end{titlepage}

    \chapter*{Аннотация}
        В данной работе рассматривается проблема классификации пакетов в рамках архитектуры сетевого процессора. В качестве критериев выбора структур данных были минимизация 
        используемой памяти и минимизация времени классификации. Был проведён обзор существующих структур данных, на основе которого было выбрано АВЛ дерево,
        с использованием алгоритма представления префиксов, как скалярных величин. На основе выбранной структуры данных было реализовано
        адаптированное, под рассматриваемую архитектуру сетевого процессора АВЛ дерево. Оно было реализовано на эмуляторе сетевого процессора.
        Также было проведено экспериментальное исследование и проведено сравнение c другими структурами данных. Использование адаптированного АВЛ дерева позволило 
        сократить количество тактов сетевого процессора на обработку одного пакета и уменьшить использование памяти вычислительных блоков конвейера.
        
    \newpage
    \tableofcontents
    \newpage

    \chapter*{Введение}
        \addcontentsline{toc}{chapter}{\protect\numberline{}Введение}%
         В настоящее время активно развивается технология программно-конфигурируемых сетей (ПКС), в которых требуются высокопроизводительные 
        коммутаторы. Возникает задача разработки программируемого сетевого процессора, являющегося основным элементом коммутаторов. Сетевой процессор 
        представляет из себя интегральную микросхему, специализированную для обработки сетевых пакетов, которая выполняет следующие функции:
        получение пакета с физического порта, выделение заголовка,
        классификация пакета по его заголовку, принятие решения о дальнейшем пути следования пакета, отправка пакета на физический порт~\cite{chao2007high:1}.
        В настоящее время активно ведётся разработка программируемых сетевых процессоров. Под программируемым сетевым процессором
        понимается сетевой процессор, который позволяет менять программу обработки пакетов и набор различаемых полей заголовков.
        
        Исходя из функций сетевого процессора целесообразно рассматривать архитектуру, основанную на
        наборе конвейеров, которая позволяет с фиксированной задержкой обрабатывать каждый пакет. 
        Конвейер в сетевом процессоре состоит из вычислительных блоков. В данной работе рассматривался этап классификации пакетов. 
        Под классификацией понимается процесс идентификации сетевого пакета по его признакам.
        Таблица классификации $-$ набор правил содержащих в себе признаки, по которым идентифицируется группа пакетов,
        и действия, которые сетевой процессор выполняет над данной группой пакетов. 
        Таким образом, для выполнения классификации сетевой процессор должен включать в себя ассоциативное устройство. Для реализации этого устройства естественным будет использование 
        ассоциативной памяти. Однако единственный контроллер ассоциативной памяти будет являться узким местом, так как к нему должны иметь доступ все стадии конвейера.
        Соответственно, возникает потребность усложнения архитектуры, например, путём добавления в неё нескольких контроллеров ассоциативной памяти.
        Чтобы избежать сложной организации памяти в архитектуре можно отказаться от использования ассоциативной памяти. 
        В таком случае одно из решений $-$ совместить память команд и данных, и разместить память на кристалле.
        Таким образом, возникает задача разработки структур данных для поиска в таблицах классификации в сетевом процессоре без выделенного ассоциативного устройства.
 

    \chapter{Цели и задачи работы}
        Целью данной работы является исследование и разработка структур данных и алгоритмов для поиска в таблицах классификации 
        в рамках архитектуры сетевого процессора без выделенного ассоциативного устройства.
        Для выполнения описанной цели ставятся следующие задачи:

        \begin{enumerate}
            \item Провести обзор структур данных для поиска в таблицах классификации пакетов с целью выбора для применения в рассматриваемой архитектуре сетевого процессора.
            \item Предложить усовершенствование и адаптацию выбранных структур данных для реализации в рамках рассматриваемой архитектуры сетевого процессора.
            \item Реализовать усовершенствованные структуры данных и алгоритмы поиска в эмуляторе сетевого процессора.
            \item Провести экспериментальное исследование реализованных структур данных.
        \end{enumerate}
       
    \chapter{Рассматриваемая архитектура сетевого процессора}
        В рамках работы важны лишь некоторые особенности рассматриваемой архитектуры, а именно, отсутствие выделенного ассоциативного устройства памяти и конвейерная архитектура. 
        Данные особенности непосредственно влияют на ограничения, предъявляемые к реализуемым структурам данных.\\
        В сетевом процессоре используется конвейерная архитектура, каждый конвейер состоит из 10 вычислительных блоков. 
        Вычислительный блок $-$ это набор более низкоуровневых устройств, которые в данной работе не рассматриваются. 
        Каждый вычислительный блок имеет доступ к участку памяти, в котором располагаются микрокод и данные.
        Существует ограничение на количество тактов, которое один пакет может обрабатываться на вычислительном блоке, оно соответствует 25 тактам.
        Также один вычислительный блок имеет доступ к 2 мегабайтам памяти
        Из-за особенностей микроархитектуры, отсутствует отдельная область памяти, в которой хранятся данные. Поэтому микрокод содержит в себе все данные,
        необходимые для классификации пакетов.

        \begin{figure}[h]
            \includegraphics[width=\textwidth]{npu_all.png}
            \caption{Архитектура рассматриваемого сетевого процессора}
        \end{figure}
        Для достижения требуемой производительности в архитектуре используется ограничение на количество тактов обработки пакета
        на одном вычислительном блоке. Каждый пакет может обрабатываться на одном вычислительном блоке не более чем 25 тактов.
        
    \chapter{Постановка задачи} 
        Разработать структуру данных для классификации пакетов в рамках рассматриваемой архитектуры сетевого процессора без выделенного ассоциативного устройства. 
        Из-за ограничений рассматриваемой архитектуры сетевого процессора к реализуемой структуре данных предъявляются следующие требования:

        \begin{itemize}
            \item Объём памяти занимаемый структурой данной данных не должен превышать 2 мегабайт.
            \item Количество тактов затраченных на поиск не должно превышать 250 тактов.
            \item Реализуемая структура данных должна быть универсальна. Под универсальностью понимается возможность структуры данных хранить и обрабатывать битовые строки длинной до 128 бит.
        \end{itemize}
    \chapter{Обзор структур данных для классификация пакетов в рамках рассматриваемой архитектуре сетевого процессора}
        Целью данного обзора является сравнение и выбор структур данных для поиска в таблицах классификации. 

        В силу особенностей архитектуры сетевого процессора, а именно отсутствия адресуемой памяти, 
        которая требуется для не древовидных структур данных будут рассматриваться только древовидные структуры данных. 
        
        В настоящем обзоре будут использоваться следующие критерии:
        \begin{enumerate}
            \item {\bf Асимптотическая сложность поиска} $-$ позволяет оценить использование ресурсов сетевого процессора для поиска в структуре данных.
            \item {\bf Универсальность структуры данных} $-$ используемая структура данных должна поддерживать поиск произвольных битовых строк,
                  не превосходящих по длине 128 бит.
            \item Дополнительные данные, необходимые для построения структуры данных.
            \item {\bf Необходимость использования адресуемой памяти} $-$ некоторым рассматриваемым структурам данных требуется адресуемая память для их реализации, соответствующей их асимптотической сложности.
            \item Количество вершин, которое необходимо посетить для поиска в случае хранения битовых строк длиной до 48 бит.
            \item Возможность оптимизации алгоритма поиска с помощью инструкций микроархитектуры ядер конвейера сетевого процессора.
            \item Необходимость изменения микроархитектуры вычислительных блоков конвейера сетевого процессора.
            \item {\bf Оценка объёма памяти, занимаемой структурой данных} $-$ рассматривается объём памяти, занимаемый структурой данных, для 10000 вхождений префиксов IPv4.
            \item Сложность добавления и удаления элементов из рассматриваемой структуры данных.
        \end{enumerate}
        Сравнение структур данных будет проводиться по критериям 1, 2, 4, 5, 8. {\ttfamily Ожидает верификации}
        \newpage
        В графическом изображении структур данных будет использован следующий набор префиксов:
        \begin{itemize}
            \item $P1 - 001*$
            \item $P2 - 1100*$
            \item $P3 - 0*$
            \item $P4 - 11*$
            \item $P5 - 101*$
            \item $P6 - 1011$
        \end{itemize}


        В графическом изображении префикс, записанный внутри вершины, обозначает, что в этой вершине дерева хранится префикс.
        Отметки ${0,1}$ на рёбрах дерева обозначают переход в следующую вершину если на данной позиции в текущей битовой строке находится бит соответствующий указанной отметке.
    \section{Метод подсчёта объёма памяти занимаемый рассматриваемой структурой данных}
        Так как в рассматриваемой архитектуре сетевого процессора совмещена память для команд и данных, объём памяти занимаемой структурой данных 
        для конкретного набора битовых строк соответствует объёму памяти занимаемому программой, которая реализует рассматриваемую структуру данных.

        Для реализуемой структуры данных вычисляется количество вершин для хранения 10000 битовых строк $-$ $N$, затем вычисляется среднее количество инструкций на 
        одну вершину $-$ $S$. Тогда искомый объём памяти занимаемый структурой данных будет рассчитываться по формуле $M = 128 * N * S$.
    \section{Структуры данных для классификации пакетов в рамках рассматриваемой архитектуры сетевого процессора}
        \subsection{Двоичное однобитное дерево}
            Наиболее распространенная $[ ]$ структура данных для поиска наиболее длинного префикса. Строится дерево по заданным префиксам
            так, что каждому биту префикса соответствует своя вершина в дереве. 
            Поиск осуществляется спуском в глубину по битам элемента, для которого выполняется поиск. 
            Поиск заканчивается тогда, когда достигнута пустая вершина, результатом поиска считается последний встретившийся префикс. $[ ]$
            На Рис. 1 изображён пример данного дерева для набора префиксов.

            \begin{figure}[h]      
                \centering 
                \includegraphics[width=0.6\textwidth]{binary.png}
                \caption{Пример построения двоичного однобитного дерева}\label{fig:mesh1}
            \end{figure}
            
            \begin{itemize}
                \item\textbf{Асимптотическая сложность поиска} для этой структуры данных соответствует {\ttfamily $O(W)$},
                где {\ttfamily $W$} $-$ максимальная длина префикса.
                \item\textbf{Универсальность структуры данных} $-$ рассматриваемое дерево может быть использовано для любых данных представимых в битовом виде.
                \item\textbf{Дополнительные данные для построения} структуры данных не требуются.
                \item\textbf{Количество вершин, которое необходимо посетить для поиска} не превосходит 48.
                %{\ttfamily (Комментарий: тут вообще имеется в виду худший случай, т.е. кода мы ищем префикс IPv6\\ длиной 128 бит. При этом он есть в дереве и есть все префиксы до него.)}\\
                \item\textbf{Оптимизацию алгоритма поиска} проводить не требуется.
                \item\textbf{Изменение микроархитектуры ядер конвейера сетевого процессора} не требуется для реализации данного алгоритма.
                \item\textbf{Оценка памяти, занимаемой структурой данных}. Будем считать, что на каждую вершину используется не более 5 инструкций,
                тогда рассматриваемая структура данных занимает не более чем 19000 Кбайт.
                \item\textbf{Добавление и удаление элементов} из рассматриваемой структуры данных имеет асимптотическую сложность 
            {\ttfamily $O(\log_2{W})$}, где {\ttfamily $W$} $-$ максимальная длина префикса и не требует перестройки всей структуры данных.\\
            \end{itemize}

        \subsection{Сжатое двоичное дерево}
            Рассматриваемая структура данных $-$ оптимизация двоичного однобитного дерева $[ ]$. Для построения данного дерева необходимо построить двоичное однобитное дерево, 
            затем провести процедуру сжатия, а именно все вершины, у которых только один лист, сокращаются, и в следующую вершину заносятся данные о 
            количестве пропущенных вершин. Таким образом, построенное дерево не имеет вершин с одним листом. Благодаря описанной оптимизации, данное дерево
            занимает меньше памяти, чем двоичное однобитное дерево. Это обусловлено отсутствием проходных вершин. Однако, уменьшается количество затраченных инструкций на поиск лишь 
            для префиксов, перед которыми были однолистные вершины. Для худшего случая, когда для префикса есть все его более короткие версии, количество вершин, 
            которые нужно посетить для поиска, аналогично двоичному однобитному дереву. $[ ]$

            \begin{figure}[h]
                \centering
                \includegraphics[width=0.6\textwidth]{compressed_binary.png}
                \caption{Пример построения сжатого двоичного дерева}\label{fig:mesh2}
            \end{figure}

            \begin{itemize}
                \item\textbf{Асимптотическая сложность поиска} для этой структуры данных соответствует {\ttfamily $O(W)$},
                где {\ttfamily $W$} $-$ максимальная длина префикса.
                \item\textbf{Универсальность структуры данных} $-$ рассматриваемое дерево может быть использовано для любых данных представимых в битовом виде.
                \item\textbf{Дополнительные данные} для построения структуры данных не требуются.
                \item\textbf{Количество вершин, которое необходимо посетить для поиска} не превосходит 48.
                \item\textbf{Оптимизацию алгоритма поиска} проводить не требуется.
                \item\textbf{Изменение микроархитектуры ядер конвейера сетевого процессора} не требуется для реализации данного алгоритма.
                \item\textbf{Оценка памяти занимаемой структурой данных} $-$ рассматриваемая структура данных занимает не более чем 1500 Кбайт.
                \item\textbf{Добавление и удаление элементов} из рассматриваемой структуры данных имеет асимптотическую сложность 
                {\ttfamily $O(\log_2{W})$}, где {\ttfamily $W$} $-$ максимальная длина префикса и не требует перестройки всей структуры данных.
            \end{itemize}
        \subsection{Мультибитное сжатое дерево}
            Данное дерево $-$ оптимизация двоичного сжатого дерева $[ ]$. Используется другая структура деревьев, когда в каждой вершине
            может быть максимум не два листа, а {\ttfamily $2^h$}, где {\ttfamily $h$} $-$ это максимальная глубина поддерева данной вершины.
            При использовании рассматриваемой структуры данных в рамках архитектуры процессора общего назначение, количество операций для поиска ограничивается глубиной дерева,
            которая равна {\ttfamily $\frac{W}{K}$}, где {\ttfamily $W$} $-$ длина максимального префикса, а {\ttfamily $K$} $-$ количество уровней в нашем дереве.
            В рамках архитектуры сетевого процессора будет рассматриваться реализация данного дерева, в которой используется линейный поиск в каждой вершине по дочерним вершинам $[ ]$.

            \begin{figure}[h]
                \centering
                \includegraphics[width=0.6\textwidth]{multybit_compressed.png}
                \caption{Пример построения мультибитного сжатого дерева}\label{fig:mesh3}
            \end{figure}

            \begin{itemize}
                \item\textbf{Асимптотическая сложность поиска} для этой структуры данных соответствует {\ttfamily $O(\frac{W}{K}*N)$},
                где {\ttfamily $K$} $-$ количество префиксов в рассматриваемой структуре данных, а {\ttfamily $W$} $-$ максимальная длина префикса
                и {\ttfamily $N$} $-$ количество листьев в вершине.
                \item\textbf{Универсальность структуры данных} $-$ рассматриваемое дерево может быть использовано для любых данных представимых в битовом виде.
                \item\textbf{Дополнительные данные} для построения структуры данных не требуются.
                \item\textbf{Количество вершин, которое необходимо посетить для поиска} не превосходит 20. 
            %    {\ttfamily Коментарий: Тут надо как-то сказать, что вершин то мы посещяем мало, а вот инструкций на каждую вершину много}
                \item\textbf{Оптимизацию алгоритма поиска} нужна, для организации быстрого доступа к листьям внутри одной вершины. 
                \item\textbf{Изменение микроархитектуры ядер конвейера сетевого процессора} для данной реализации не требуется.
                \item\textbf{Оценка памяти занимаемой структурой данных} $-$ Будем считать, что каждая вершина занимает 15 инструкций в среднем, из-за линейного поиска по листьям. 
                Тогда рассматриваемая структура данных занимает не более чем 25000 Кбайт.
                \item\textbf{Добавление и удаление элементов} из рассматриваемой структуры данных имеет асимптотическую сложность 
                {\ttfamily $O(\log_2{W})$}, где {\ttfamily $W$} $-$ максимальная длина префикса и не требует перестройки всей структуры данных.\\
            \end{itemize}

            При реализации микроинструкции, которая по заданной метке перехода и маске переходит на инструкцию находящуюся на метке + значение регистра с применённой маской,
            есть возможность реализовать структуру данных, в которой алгоритм поиска будет быстрее. Рассмотрим эту структуру данных в рамках архитектуры с добавлением указанной микроинструкции.\\
            \begin{itemize}
                \item\textbf{Асимптотическая сложность поиска} для этой структуры данных соответствует {\ttfamily $O(\frac{W}{K})$},
                где {\ttfamily $K$} $-$ количество префиксов в рассматриваемой структуре данных, а {\ttfamily $W$} $-$ максимальная длина префикса.
                \item\textbf{Универсальность структуры данных} $-$ рассматриваемое дерево может быть использовано для любых данных представимых в битовом виде.
                \item\textbf{Дополнительные данные} для построения структуры данных не требуются.
                \item\textbf{Количество вершин, которое необходимо посетить для поиска} не превосходит 20.
                \item\textbf{Оптимизацию алгоритма поиска} нужна, для организации быстрого доступа к листьям внутри одной вершины.
                \item\textbf{Изменение микроархитектуры ядер конвейера сетевого процессора} требуется аппаратная реализации быстрого перехода внутри одной вершины.
                А именно необходима инструкция, которая по заданной метке перехода и маске переходит на инструкцию находящуюся на метке + значение регистра с применённой маской.
                \item\textbf{Оценка памяти занимаемой структурой данных} $-$ Будем считать, что каждая вершина занимает 5 инструкций, благодаря новой микроинструкции.
                Тогда рассматриваемая структура данных занимает не более чем 1800 Кбайт.
                \item\textbf{Добавление и удаление элементов} из рассматриваемой структуры данных имеет асимптотическую сложность 
                {\ttfamily $O(\log_2{W})$}, где {\ttfamily $W$} $-$ максимальная длина префикса и не требует перестройки всей структуры данных.\\
            \end{itemize}

        \subsection{Бинарный поиск по длинам префиксов}
            Структура данных основана на построении специальных таблиц для префиксов определённой длины $[ ]$. Пусть максимальная длина префикса {\ttfamily $W$}, 
            тогда строятся таблицы {\ttfamily $h_{1},\ldots,h_{w}$}. В каждой из них хранятся префиксы длины соответствующие номеру этой таблицы. Предполагается, 
            что в каждой такой таблице реализована своя хеш-функция, которая быстро позволяет найти вхождение префикса в данную таблицу.
            Таким образом мы можем выполнить бинарный поиск по длине префиксов. В рамках рассматриваемой архитектуры сетевого процессора, реализация таких таблиц возможна только
            с использованием древовидных структур данных. $[ ]$

            \begin{figure}[h]
                \centering
                \includegraphics[width=\textwidth]{binary_search.png}
                \caption{Пример бинарного поиска по длинам префиксов}\label{fig:mesh4}
            \end{figure}

            \begin{itemize}
                \item\textbf{Асимптотическая сложность поиска для этой структуры данных} соответствует {\ttfamily $O(K*\log_2{N})$},
                где {\ttfamily $K$} $-$ количество префиксов в рассматриваемой структуре данных, а {\ttfamily $W$} $-$ максимальная длина префикса.
                и {\ttfamily $N$} $-$ количество префиксов в таблице.
                \item\textbf{Универсальность структуры данных} $-$ рассматриваемая структура данных может быть использовано только для поиска префиксов.
                \item\textbf{Дополнительные данные} для построения структуры данных не требуются.
                \item\textbf{Количество вершин, которое необходимо посетить для поиска} не превосходит 48.
                \item\textbf{Оптимизацию алгоритма поиска} требуется, а именно необходима для реализации поиска внутри таблицы.
                \item\textbf{Изменение микроархитектуры ядер конвейера сетевого процессора} требуется аппаратная реализация хеш-функций.
                \item\textbf{Оценка памяти занимаемой структурой данных} $-$ рассматриваемая структура данных занимает не более чем 20000 Кбайт.
                \item\textbf{Добавление и удаление элементов} из рассматриваемой структуры данных требует перестройки всей структуры данных.
            \end{itemize}

        \subsection{АВЛ дерево}
            Представление префиксов как скалярных префиксов позволяет использовать больший набор структур данных. $[ ]$ 
            В качестве примера рассмотрим АВЛ дерево, основной особенностью которого является правило его построения: у каждой вершины разность 
            глубины левого и правого поддерева не превосходит 1, что даёт асимптотическую сложность поиска {\ttfamily $O(1+\log_2{N})$}, 
            где {\ttfamily $N$} $-$ количество префиксов в нашей структуре данных. Из этого следует, что время поиска не зависит от длины искомых данных,
            а значит с помощью данной структуры данных очень эффективно выполнять поиск префиксов IPv6. $[ ]$
            \\
            \begin{figure}[ht]
                \centering
                \includegraphics[width=0.6\textwidth]{avl_tree.png}
                \caption{Пример построения АВЛ дерева}\label{fig:mesh5}
            \end{figure}
            \begin{itemize}
                \item\textbf{Асимптотическая сложность поиска для этой структуры данных} соответствует {\ttfamily $O(1 + \log_2{N})$},
                где {\ttfamily $N$} $-$ количество префиксов в таблице.
                \item\textbf{Универсальность структуры данных} $-$ рассматриваемое дерево может быть использовано для любых данных представимых в битовом виде.
                \item\textbf{Дополнительные данные для построения структуры данных} не требуются.
                \item\textbf{Количество вершин, которое необходимо посетить для поиска} не превосходит 19.
                \item\textbf{Оптимизацию алгоритма поиска} проводить не требуется.
                \item\textbf{Изменение микроархитектуры ядер конвейера сетевого процессора} не требуется.
                \item\textbf{Оценка памяти занимаемой структурой данных} $-$ рассматриваемая структура данных занимает не более чем 1000 Кбайт.
                \item\textbf{Добавление и удаление элементов из рассматриваемой структуры данных} имеет логарифмическую сложность, и не требует перестройки всей структуры данных.
            \end{itemize}

    \section{Сравнение структур данных}
        Для сравнения структур данных рассмотрим Таблицу 4.1 сравнения по критериям: асимптотическая сложность, универсальность структуры данных,
        количество вершин, которое необходимо посетить для поиска и оценка памяти занимаемой структурой данных. 
        
        \begin{table}[ht]
            \begin{tabular}{|m{3cm}|m{2.5cm}|m{3.5cm}|m{4cm}|m{3.2cm}|}
                \hline
                \bf Название структуры данных     & \bf Сложность поиска & \bf Универсальность & \bf Количество вершин необходимых для поиска & \bf Память, Кбайт \\
                \hline
                Двоичное однобитное дерево & $O(W)$ & да & 48 & 19000 \\
                \hline
                Двоичное сжатое дерево & $O(W)$ & да & 48 & 1500 \\
                \hline
                Мультибитное сжатое дерево & $O(\frac{W}{K}*L)$ & да & 20 & 25000 \\
                \hline
                Бинарный поиск по длинам префиксов & $O(K*\log_2{N})$ & нет & 48 & 20000 \\
                \hline
                АВЛ дерево & $O(1 + \log_2{N})$ & да & 19 & 1000 \\
                \hline
            \end{tabular}
            \caption{Сравнение структур данных; $W$ $-$ максимальная длина префикса, $N$ $-$ количество префиксов в структуре данных,
            $L$ $-$ количество дочерних вершин, $К$ $-$ глубина дерева.}
        \end{table}
        
        У каждой рассмотренной структуры данных есть свои достоинства и недостатки, рассмотрим их:
        \begin{enumerate}
            \item \textbf{Двоичное однобитное дерево} $-$ данная структура проста в реализации, но занимает много памяти и поиск требует прохождения 48 вершин.
            \item \textbf{Двоичное сжатое дерево} $-$ занимает меньше памяти, чем двоичное однобитное дерево, но поиск требует прохождения 48 вершин. 
                Соответственно использование данного дерева предпочтительнее, чем двоичного однобитного дерева.
            \item \textbf{Мультибитное сжатое дерево} $-$ занимает много памяти, но поиск требует прохождения сильно меньшего количества вершин. 
                Из-за проблем с реализацией в рамках рассматриваемой архитектуры, эта структура данных не может быть реализована.
            \item \textbf{Бинарный поиск по длинам префиксов} $-$ занимает много памяти, и может быть использован только для поиска наиболее длинного префикса.
                Также из-за проблем с реализацией на рассматриваемой архитектуре, данная структура не подходит для решения проблемы.
            \item \textbf{АВЛ дерево} $-$ занимает меньше всего памяти, и при этом, поиск требует прохождения малого количества вершин,
                которое зависит от количества вхождений в структуру данных, а не от конкретного префикса.
        \end{enumerate}

        Таким образом на основе обзора для дальнейшей реализации были выбраны две структуры данных: 
        АВЛ дерево с алгоритмом представления префиксов как скалярных величин и бинарное сжатое дерево.
    \section{Выводы}
        В данном разделе были рассмотрены следующие структуры данных: двоичное однобитное дерево, двоичное сжатое дерево, мультибитное сжатое дерево, бинарный поиск по длинам префиксов и АВЛ дерево.
        Обзор проводился с целью выбора структур данных для дальнейшей адаптации и усовершенствования в рамках рассматриваемой архитектуры сетевого процессора, 
        и дальнейшей реализации полученных структур данных. По итогам проведённого обзора были выбраны следующие структуры данных: АВЛ дерево и двоичное сжатое дерево.

        \chapter{Разработка и адаптация структур данных}
            \section{Двоичное сжатое дерево}
                \subsection{Алгоритм построения}
                \subsection{Предложенные оптимизации}
            \section{АВЛ дерево}
                \subsection{Алгоритм представления префиксов как скалярных величин}
                \subsection{Алгоритм построения дерева}
                \subsection{Предложенные оптимизации}
            \section{Выводы}
        \chapter{Реализация разработанных структур данных}
            \section{Эмулятор сетевого процессора}
            \section{Предложенные изменения эмулятора сетевого процессора}
        \chapter{Экспериментально исследование реализованных структур данных}
            \section{Методика экспериментального исследования}
            \section{Двоичное сжатое дерево}
            \section{АВЛ дерево}
            \section{Выводы}
        \chapter{Заключение}

\printbibliography{}
\addcontentsline{toc}{chapter}{\protect\numberline{}Список Литературы}%

\end{document} 
